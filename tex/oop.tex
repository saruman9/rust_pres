\begin{frame}{\insertsubsubsection}
  \center%
  \includegraphics<1>[height = 0.9\textheight]{oop_problems.jpg}
  \includegraphics<2>[height = 0.9\textheight]{oop_my_problems.jpg}

  \note<1>{

    В C++ есть классы, в Python есть классы, во \textbf{множестве языков есть
      парадигма ООП}, а это объекты, методы, поля и т. п., а если быть точным:
    \textbf{инкапсуляция, наследование, полиморфизм} и другие концепции. Лично
    мне парадигма \textbf{ООП приносила много проблем}, я её не понимал, считал
    её \textbf{неудобной и ущербной}.

  }

  \note<2>{

    Я читал много книг по ООП и \textbf{пытался понять, как нужно писать
      правильно}. Одно время даже на Smalltalk хотел программировать, чтобы всё
    понять как следует. Из книг я понял одно, что \textbf{не ООП плох, а я туп}.

    Кто до конца понял ООП и считает, что это именно та парадигма
    программирования, которой \textbf{стоит придерживаться}?

    В Rust же \textbf{чистого ООП}, которые привыкли многие видеть, ---
    \textbf{нет}. Да, там есть некоторые концепции, например полиморфизм,
    инкапсуляция, но, например, чистого наследования там как такового нет.
    \textbf{Там другой подход}, который мне как раз таки пришёлся по душе.
    Конечно, это всё дело вкуса, но мало ли, кому-то не нравится ООП парадигма,
    посмотрите на Rust или на функциональные языки, потому что даже на C++ можно
    писать в другой парадигме.

    Из-за того, что в Rust нет чистого ООП многие \textbf{люди приходят в язык},
    пытаются \textbf{писать не идиоматичный код}, \textbf{компилятор бьёт} их
    больно по рукам, \textbf{они плачут и ругают Rust}.

  }
\end{frame}