\begin{frame}[fragile]{\insertsubsection}
  \begin{minted}[gobble = 4, frame = lines, framesep = 7pt, label = C,
    linenos]{c}
    memset(c, 0, sizeof(Net_Crypto));
    free(c);
  \end{minted}

  \note{

    После C, Rust --- это как глоток свежего воздуха. Но одним из самых ярких
    плюсов для меня считается \textbf{отсутствие UB}.

    В примере \texttt{memset} может затереться, потому что компилятор думает: а
    зачем занулять, если я сейчас всё равно удалю указатель. \textbf{Это
      реальный пример} из криптографической библиотеки, которая была
    \textbf{переписана на Rust} впоследствии.

    В Rust есть только \textbf{Unsound} штуки, по-другому --- баги в компиляторе
    или в стандартной библиотеке, которые \textbf{естественно фиксятся}.
    \textbf{UB нет в RFC}, в отличие от \textbf{стандарта C/C++}, в котором чёрт
    ногу сломит.

  }

\end{frame}
