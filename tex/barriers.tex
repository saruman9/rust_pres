\begin{frame}{\insertsubsection}
  \begin{columns}
    \begin{column}{.5\textwidth}
      Typically scope:
      \begin{itemize}
      \item Object-oriented programming
      \item Garbage collected programming language
      \item Dynamic programming language
      \end{itemize}
    \end{column}

    \begin{column}{.4\textwidth}
      Rust scope:
      \begin{itemize}
      \item No object-oriented programming
      \item No garbage collector
      \item No dynamic typing
      \end{itemize}
    \end{column}
  \end{columns}

  \note{

    Изучать \textbf{новое всегда сложно} и времязатратно, а в Rust для типичного
    программиста будет \textbf{много всего нового}. По сути придётся изучать
    \textbf{не просто новый язык} программирования (что уже само по себе не
    легко), а также \textbf{статическую систему типов}, понятия \textbf{стека и
      кучи}, понятие \textbf{времени жизни объекта}, \textbf{дженерики},
    \textbf{алгебраические типы}, \textbf{отвыкнуть от ООП}, немного погрузиться
    в \textbf{функциональное программирование} и т. д.

  }
\end{frame}
