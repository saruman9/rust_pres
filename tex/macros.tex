\begin{frame}[fragile]{\insertsubsection}
  \begin{onlyenv}<1>
    \begin{minted}[fontsize = \footnotesize, gobble = 4, frame = lines, framesep
      = 7pt, linenos, breaklines, label = Declarative macros]{rust}
    #[macro_export]
    macro_rules! vec {
        ( $( $x:expr ),* ) => {
          {
            let mut temp_vec = Vec::new();
            $(temp_vec.push($x);)*
            temp_vec
          }
        };
    }
    let vec_int = vec!(1, 2, 3, 4);
    let vec_str = vec!("H", "a", "c", "k", "e", "r");
    println!("{:?} {:#?}", vec_int, vec_str);
  \end{minted}
  \end{onlyenv}

  \begin{onlyenv}<2>
  \begin{minted}[gobble = 4, frame = lines, framesep = 7pt, linenos, breaklines,
    label = Procedural macros]{rust}
    #[route(GET, "/")]
    fn index() {
      ...
    }
  \end{minted}
  \end{onlyenv}

  \note<1>{

    В Rust \textbf{макросы --- это очень мощный инструмент}. Конечно, они не
    такие, как в Lisp, но как по мне --- не менее мощные.

    В Rust два типа макросов --- \textbf{декларативные и процедурные}.

    Пример декларативного --- такой же, как в C, только \textbf{основаны они на
      AST} представлении, отчего ошибки в них может \textbf{отловить
      компилятор}, они \textbf{гигиенически чистые}, как в Scheme.

  }

  \note<2>{

    Также есть процедурные макросы, которые в свою очередь делятся на:
    \begin{itemize}
    \item \texttt{\#[derive]} макросы,
    \item макросы-атрибуты,
    \item макросы-функции.
    \end{itemize}

    Представлен пример макроса-атрибута, позволяющих сократить размеры кода в
    разы.

  }
\end{frame}
