\begin{frame}{\insertsubsection}

  \center%
  \includegraphics<1>[height = .8\textheight]{so_compiler.jpg}%
  \includegraphics<2>[height = .8\textheight]{runtime_error.jpg}%

  \note<1>{

    Rust похож \textbf{на Haskell системой комплексных типов}, поэтому многие
    \textbf{ошибки выявляются на этапе компиляции}.

    Отсюда как раз таки и вытекает \textbf{минус Rust по версии многих
      новичков}. \textbf{Компилятор очень больно и много бьёт} по рукам. Многие
    жалуются, что они делают всё правильно, а компилятор всё равно ругается,
    особенно это касается момента разработки, когда требуется расставить всё
    \textbf{время жизни объектов}. Это называется \textbf{борьба с borrow
      checker}.

  }

  \note<2>{

    Вообще, Rust такой язык, что \textbf{если программа скомпилировалась без
      ошибок}, это значит, что в 98\% случаев программа будет работать
    \textbf{без ошибок в runtime}. И у вас не возникнет такой ситуации, как на
    слайде.

  }

\end{frame}
