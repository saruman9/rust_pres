\begin{frame}{\insertsubsection}

  \note{

    Rust похож на Haskell системой комплексных типов, поэтому многие ошибки
    выявляются на этапе компиляции. Вообще, Rust такой язык, что если программа
    скомпилировалась без ошибок, это значит, что в 98\% случаев программа будет
    работать без ошибок в runtime.

    Отсюда как раз таки и вытекает минус Rust по версии многих новичков.
    Компилятор очень больно и много бьёт по рукам. Многие жалуются, что они
    делают всё правильно, а компилятор всё равно ругается, особенно это касается
    момента разработки, когда требуется расставить всё время жизни объектов. Это
    называется борьба с borrow checker.

  }

\end{frame}
