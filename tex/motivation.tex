\begin{frame}{Motivation}

  \center%
  \includegraphics<1>[width = 0.9\textwidth]{c_p_p_die.jpg}
  \includegraphics<2>[height = 0.9\textheight]{rust_evangelism.png}

  \note<1>{

    В последнее время \textbf{всё чаще} в IT новостях можно услышать, что был
    разработан тот или иной \textbf{проект на Rust}, будь то Web (WebAssembly)
    или Embedded (TrustZone, IoT) или GameDev или консольная утилита или новая
    ОС или файловая система или кодек или декодировщик и т. д.

    Интерес к языку среди программистов растёт всё больше и больше. Вот уже и
    \textbf{среди безопасников интерес также подогревается}, не смотря на то,
    что многие относятся скептически. Я даже \textbf{в исследовательском центре}
    всё чаще сталкиваюсь с проектами на Rust: при изучении ChromeBook (система
    виртуализации, песочница), блокчейн проектов (Rust здесь практически
    лидирует), изучение новых фаззеров (на Rust), в IoT устройствах.

  }
      

  \note<2>{

    Я отчасти Rust евангелист (в принципе как и евангелист Linux и KISS
    принципа), поэтому буду призывать попробовать данный язык по возможности.
    Статистика показывает, что \textbf{люди, которые перешагнули определённый
      порог} при изучении, \textbf{получают удовольствие} от того, что
    разрабатывают на Rust.

    Мне интересны языки программирования, поэтому я знаю, что существует ещё
    множество других интересных и полезных языков (Racket, Haskell, Ada, OCaml,
    SPARK, Coq, Nim и др.), поэтому выбирайте язык исходя из задачи, а не
    задачу, исходя из известного вам языка.

    В своём докладе я сделаю упор на сам язык, а не на его аудит, т. к. на
    сколько мне известно, до сих пор небольшой процент людей нашей компании
    знаком с данным языком. Это недоразумение я и хочу исправить.
      
  }
\end{frame}
