\begin{frame}[fragile]{\insertsubsubsection}
  \begin{onlyenv}<1>
  \begin{minted}[gobble = 4, frame = lines, linenos, breaklines, label = Result]{rust}
    enum Result<T, E> {
      Ok(T),
      Err(E),
    }
  \end{minted}
  \end{onlyenv}

  \begin{onlyenv}<2>
  \begin{minted}[gobble = 4, frame = lines, linenos, breaklines, label = Result]{rust}
    pub fn hack_program(program: &Program) -> Result<Shell> {...}

    match hack_program(&program) {
      Ok(shell) => connect(shell),
      Err(error) => {
        // Do something with error
      }
    }
  \end{minted}
  \end{onlyenv}

  \begin{onlyenv}<3>
  \begin{minted}[gobble = 4, frame = lines, linenos, breaklines, label = Result]{rust}
    fn hack_world(world: World) -> Result<Power, u32> {
      hack_program(&program)?;

      for program in &world.programs() {
        hack_program(program).map(install_spy).map(get_money)?;
      }
    }

  \end{minted}
  \end{onlyenv}

  \note<1>{

    Некоторые ошибки могут быть не на столько критичными, чтобы паниковать,
    поэтому вводится \textbf{новый алгебраический тип}, который может содержать
    в себе либо \textbf{возвращаемый объект} (в случае успеха), либо
    \textbf{объект-ошибку}.

    Это удобно, т. к. позволяет нам узнать \textbf{полную информацию об ошибке}
    в случае чего, а также задавать абсолютно \textbf{любые типы в качестве
      возвращаемых}.

  }

  \note<2>{

    Вернёмся к нашей \textbf{крутой программе}, которая ломает всё, что плохо
    лежит.

    Когда мы пытаемся взломать программу, мы \textbf{можем получить ошибку}, а
    \textbf{можем получить шелл}. В случае, если получаем шелл, то подключаемся,
    если же взломать не получилось, то либо пытаемся понять почему
    (\textbf{обрабатывая ошибку}), либо \textbf{пробрасываем ошибку} дальше,
    либо \textbf{завершаем программу} и грустим, либо всё, \textbf{что вам
      угодно}.
    
  }

  \note<3>{

    Для \textbf{пробрасывания ошибки} мы можем использовать \textbf{?}.

    Также мы можем \textbf{создавать цепочки вызовов функций}, которые будут
    выполняться в случае успеха.

    Можно ещё много всего показывать и рассказывать, но \textbf{времени нет}.

  }

\end{frame}