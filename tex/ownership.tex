{

  \setbeamercolor{background canvas}{bg=}%
  \includepdf[pages = 1]{o_and_b.pdf}%

  \note {

    Для того, чтобы понять, \textbf{почему Rust} считается безопасным языком,
    нужно понимать, \textbf{как он работает}, какие концепции используются.

    Сейчас будет большой блок, он может показаться \textbf{долгим и скучным}, но
    советую \textbf{внять ему}, потому что это ядро Rust.

    Мне было лень, поэтому я взял хорошо оформленные и понятные слайды
    презентации Николаса Матсакиса (один из главных разработчиков Rust), за что
    ему большое спасибо.

  }

}

{

  \setbeamercolor{background canvas}{bg=}%
  \includepdf[pages = 38]{o_and_b.pdf}%

  \note {

    \textbf{Владение и заимствование} --- вот одни из главных концепций ЯП Rust.

    На следующих слайдах я расскажу, что это такое.

  }

}

{

  \setbeamercolor{background canvas}{bg=}%
  \includepdf[pages = 39]{o_and_b.pdf}%

  \note {

    И так, концепция \textbf{владения}.

    У нас есть книга, мы ею \textbf{владеем}.

  }

}

{

  \setbeamercolor{background canvas}{bg=}%
  \includepdf[pages = 40]{o_and_b.pdf}%

  \note {

    Как только \textbf{мы передаём} книгу, мы \textbf{перестаём ей владеть}. У
    книги появляется новый хозяин.

  }

}

{

  \setbeamercolor{background canvas}{bg=}%
  \includepdf[pages = 41]{o_and_b.pdf}%

  \note {

    Рассмотрим на примере кода.

    У нас есть переменная \texttt{name} (\textbf{наша книга}) типа
    \texttt{String}.

  }

}

{

  \setbeamercolor{background canvas}{bg=}%
  \includepdf[pages = 42]{o_and_b.pdf}%

  \note {

    Также у нас есть функция \texttt{helper} --- будущий хозяин нашей строки
    (книги). Что же происходит при вызове функции?

  }

}

{

  \setbeamercolor{background canvas}{bg=}%
  \includepdf[pages = 43]{o_and_b.pdf}%

  \note {

    Посмотрим на определение функции \texttt{helper} --- в качестве параметра
    она принимает объект с типом \texttt{String}. Она принимает \textbf{не
      ссылку} и \textbf{не само значение}, а именно возможность владения данным
    \textbf{объектом}, если можно так сказать.

  }

}

{

  \setbeamercolor{background canvas}{bg=}%
  \includepdf[pages = 44]{o_and_b.pdf}%

  \note {

    Возвращаемся обратно, \textbf{вызываем функцию}.

  }

}

{

  \setbeamercolor{background canvas}{bg=}%
  \includepdf[pages = 45]{o_and_b.pdf}%

  \note {

    При вызове функции у нас \textbf{происходит смена владельца} нашей книги.

  }

}

{

  \setbeamercolor{background canvas}{bg=}%
  \includepdf[pages = 46-47]{o_and_b.pdf}%

  \note {

    После того, как \textbf{функция отработала} --- она, то бишь хозяин,
    \textbf{исчезает}.

  }

}

{

  \setbeamercolor{background canvas}{bg=}%
  \includepdf[pages = 48]{o_and_b.pdf}%

  \note {

    А если \textbf{исчезает хозяин}, то вместе и \textbf{с ним объект}, которым
    он владел. Это называется \textbf{RAII} (Resource acquisition is
    initialization) --- \textbf{получение ресурса есть инициализация}.

  }

}

{

  \setbeamercolor{background canvas}{bg=}%
  \includepdf[pages = 49]{o_and_b.pdf}%

  \note {

    Если мы попытаемся \textbf{ещё раз вызвать функцию}, которая в качестве
    аргумента принимает объект, которого \textbf{у нас уже нет}, т. к. мы его
    уже отдали, \textbf{что же тогда произойдёт}?

  }

}

{

  \setbeamercolor{background canvas}{bg=}%
  \includepdf[pages = 50]{o_and_b.pdf}%

  \note {

    Правильно, \textbf{ошибка} --- использование уже перемещённого значения.

  }

}

{

  \setbeamercolor{background canvas}{bg=}%
  \includepdf[pages = 51]{o_and_b.pdf}%

  \note {

    Посмотрим как данная концепция (или \textbf{её подобие}) реализована в
    других языках.

    У нас также есть объект, но типа \texttt{Vector} (что по сути также может
    быть строкой, вектором символов).

  }

}

{

  \setbeamercolor{background canvas}{bg=}%
  \includepdf[pages = 52]{o_and_b.pdf}%

  \note {

    У нас есть функция, которая также в качестве параметра принимает объект типа
    \texttt{Vector}.

  }

}

{

  \setbeamercolor{background canvas}{bg=}%
  \includepdf[pages = 53]{o_and_b.pdf}%

  \note {

    Но в данном случае параметр функции передаётся \textbf{в качестве ссылки}.

  }

}

{

  \setbeamercolor{background canvas}{bg=}%
  \includepdf[pages = 54]{o_and_b.pdf}%

  \note {

    Снова возвращаемся и вызываем нашу функцию.

  }

}

{

  \setbeamercolor{background canvas}{bg=}%
  \includepdf[pages = 56]{o_and_b.pdf}%

  \note {

    В \textbf{Java} при вызове функции у нас оказывается два хозяина одного
    объекта.

  }

}

{

  \setbeamercolor{background canvas}{bg=}%
  \includepdf[pages = 57]{o_and_b.pdf}%

  \note {

    Хорошо. Продолжаем выполнение функции.

  }

}

{

  \setbeamercolor{background canvas}{bg=}%
  \includepdf[pages = 58]{o_and_b.pdf}%

  \note {

    Функция закончила своё выполнение.

  }

}

{

  \setbeamercolor{background canvas}{bg=}%
  \includepdf[pages = 59]{o_and_b.pdf}%

  \note {

    Функция исчезает, как и в Rust. Но \textbf{что же происходит с объектом}?

  }

}

{

  \setbeamercolor{background canvas}{bg=}%
  \includepdf[pages = 60]{o_and_b.pdf}%

  \note {

    Объект никуда не пропадает, он \textbf{остаётся у своего первоначального
      хозяина}.

    Также \textbf{возможен повторный вызов} функции \texttt{helper}.

  }

}

{

  \setbeamercolor{background canvas}{bg=}%
  \includepdf[pages = 61]{o_and_b.pdf}%

  \note {

    Главная функция заканчивает свою работу и также исчезает.

  }

}

{

  \setbeamercolor{background canvas}{bg=}%
  \includepdf[pages = 62]{o_and_b.pdf}%

  \note {

    При этом \textbf{созданный функцией объект остаётся}. \textbf{Для чего это
      нужно?}

  }

}

{

  \setbeamercolor{background canvas}{bg=}%
  \includepdf[pages = 63]{o_and_b.pdf}%

  \note {

    Представим, что функция \texttt{helper} внутри себя создаёт поток.

  }

}

{

  \setbeamercolor{background canvas}{bg=}%
  \includepdf[pages = 64]{o_and_b.pdf}%

  \note {

    Тогда хозяином объекта, даже после исчезновения главной функции, будет
    функция \texttt{helper}.

  }

}

{

  \setbeamercolor{background canvas}{bg=}%
  \includepdf[pages = 65]{o_and_b.pdf}%

  \note {

    Когда поток закончит свою работу, то текущий хозяин объекта также перестанет
    существовать.

  }

}

{

  \setbeamercolor{background canvas}{bg=}%
  \includepdf[pages = 66]{o_and_b.pdf}%

  \note {

    Ну а сборщик мусора сделает своё дело...

  }

}

{

  \setbeamercolor{background canvas}{bg=}%
  \includepdf[pages = 67]{o_and_b.pdf}%

  \note {

    ... и удалит уже никому ненужный объект.

  }

}

{

  \setbeamercolor{background canvas}{bg=}%
  \includepdf[pages = 68]{o_and_b.pdf}%

  \note {

    Рассмотрим одну интересную особенность в Rust при передаче права владения
    объектом, а именно концепцию \textbf{клонирования}.

  }

}

{

  \setbeamercolor{background canvas}{bg=}%
  \includepdf[pages = 69]{o_and_b.pdf}%

  \note {

    Всё также вызываем нашу функцию \texttt{helper}.

  }

}

{

  \setbeamercolor{background canvas}{bg=}%
  \includepdf[pages = 70]{o_and_b.pdf}%

  \note {

    Но при передаче прав владения объектом \textbf{мы делаем клон} этого самого
    объекта!

  }

}

{

  \setbeamercolor{background canvas}{bg=}%
  \includepdf[pages = 72]{o_and_b.pdf}%

  \note {

    То есть в функцию мы отдаём не сам объект, а его клон.

  }

}

{

  \setbeamercolor{background canvas}{bg=}%
  \includepdf[pages = 73]{o_and_b.pdf}%

  \note {

    Вызываем функцию.

  }

}

{

  \setbeamercolor{background canvas}{bg=}%
  \includepdf[pages = 74]{o_and_b.pdf}%

  \note {

    После того, как функция отработает, \textbf{она исчезает} и исчезает
    аргумент, как было показано ранее, но при этом \textbf{исчезает клон
      объекта}, а не сам объект!

  }

}

{

  \setbeamercolor{background canvas}{bg=}%
  \includepdf[pages = 75]{o_and_b.pdf}%

  \note {

    После этого мы повторно можем вызвать функцию \texttt{helper}, но передать
    уже оригинальный объект.

  }

}

{

  \setbeamercolor{background canvas}{bg=}%
  \includepdf[pages = 76]{o_and_b.pdf}%

  \note {

    В этом случае, мы, конечно же, его уже теряем.

  }

}

{

  \setbeamercolor{background canvas}{bg=}%
  \includepdf[pages = 77]{o_and_b.pdf}%

  \note {

    Есть один небольшой нюанс в концепции клонирования --- \textbf{копирование}
    или по-другому \textbf{авто-клонирование}.

  }

}

{

  \setbeamercolor{background canvas}{bg=}%
  \includepdf[pages = 78]{o_and_b.pdf}%

  \note {

    Если тип объекта \textbf{обладает свойством копирования (реализован типаж)},
    то при передаче прав владения \textbf{клонирование происходит
      автоматически}.

  }

}

{

  \setbeamercolor{background canvas}{bg=}%
  \includepdf[pages = 79]{o_and_b.pdf}%

  \note {

    Вызовем функцию \texttt{helper} в первый раз, передав при этом целочисленный
    тип \texttt{i32}, который обладает свойством \texttt{Copy}.

  }

}

{

  \setbeamercolor{background canvas}{bg=}%
  \includepdf[pages = 80]{o_and_b.pdf}%

  \note {

    Произойдёт автоматическое клонирование при передаче прав владения.

  }

}

{

  \setbeamercolor{background canvas}{bg=}%
  \includepdf[pages = 81]{o_and_b.pdf}%

  \note {

    После исполнения функции объект исчезает.

  }

}

{

  \setbeamercolor{background canvas}{bg=}%
  \includepdf[pages = 82]{o_and_b.pdf}%

  \note {

    При втором вызове функции \texttt{helper} происходит то же самое, то есть
    \textbf{объект остаётся у функции-создателя}.

  }

}

{

  \setbeamercolor{background canvas}{bg=}%
  \includepdf[pages = 83]{o_and_b.pdf}%

  \note {

    Подытожим.

    У нас есть \textbf{некопируемые объекты}, при передаче прав владения
    которых, \textbf{передаётся сам объект}.

    У нас есть \textbf{объекты с возможностью клонирования}, для которых мы
    вручную можем задать, чтобы передаче прав владения \textbf{создавался клон
      объекта}.

    И у нас есть объекты, которые имеют \textbf{свойство \texttt{Copy}}. Данные
    объекты \textbf{прозрачно для разработчика копируются} при передаче в
    качестве аргумента.

  }

}