\begin{frame}{\insertsubsection}
  \begin{columns}
    \begin{column}{0.45\textwidth}
      \begin{itemize}
      \item<1-> Return code (\texttt{C}, \texttt{Go})
      \item<2-> Exceptions (\texttt{C++}, \texttt{Python})
      \item<3-> Global variable (custom)
      \item<4-> Design by Contract (\texttt{SPARK})
      \item<5-> Error (success) indicator (\texttt{Haskell})
      \end{itemize}
    \end{column}
    \begin{column}{0.6\textwidth}
      \only<1-2>{%
        \center%
        \includegraphics<1>[width = \textwidth]{go_err.jpg}%
        \includegraphics<2>[width = \textwidth]{exceptions.jpg}%
      }
    \end{column}
  \end{columns}

  \note<1>{

    Древний способ отлова ошибок, но кто-то до сих пор внедряет его в новые
    языки. Никаких гарантий, что ты \textbf{не забудешь написать проверку}, к
    тому же нужно \textbf{знать все коды ошибок}.
    
  }

  \note<2>{

    Самый популярный на данное время способ отлова ошибок.

    Меня смущает один факт --- как вы узнаете, \textbf{какой может быть
      exception} и \textbf{вызывает ли эта функция exception}?

  }

  \note<3>{

    Задаётся глобальная переменная, в которую записываются все ошибки.
    
    Встречал только в реализациях различных фреймворков, особенно на JS.
    
  }

  \note<4>{

    Проверка входных и выходных параметров заданному условию.

    Крутой способ, конечно, но это уже по части языка SPARK или контрактного
    программирования.
    
  }

  \note<5>{

    Это уже наш вариант.

  }
\end{frame}