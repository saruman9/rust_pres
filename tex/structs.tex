\begin{frame}[fragile]{\insertsubsubsection}
  \begin{minted}[gobble = 4, frame = lines, linenos, breaklines]{rust}
    struct Hacker {
      nickname: String,
      scope: Scope,
      cves: Vec<u32>,
    }

    enum Scope {
      Fuzzing,
      Developing,
      Exploiting,
      Reversing,
    }
  \end{minted}

  \note{

    В Rust есть структуры, если хотите, можете думать, что это что-то типа
    классов. Но это не классы, это всего лишь абстракция над данными.

  }

\end{frame}

\begin{frame}[fragile]{Construction}
  \begin{onlyenv}<1>
  \begin{minted}[gobble = 4, frame = lines, linenos, breaklines]{rust}
    impl Hacker {
      fn new(nickname: String, scope: Scope) -> Hacker {
        Hacker {
          nickname: nickname,
          scope: scope,
          cves: Vec::new(),
        }
      }
    }
  \end{minted}
  \end{onlyenv}
  \begin{onlyenv}<2>
  \begin{minted}[gobble = 4, frame = lines, linenos, breaklines]{rust}
    impl Hacker {
      fn new(nickname: String, scope: Scope) -> Self {
        Hacker {
          nickname, scope,
          cves: Vec::new(),
        }
      }
    }
  \end{minted}
  \end{onlyenv}

  \note<1>{

    Задать создание структуры можно таким образом, если хотите, можете называть
    это конструктором, так будет проще.

  }

  \note<2>{

    Более компактная версия.

  }
\end{frame}

\begin{frame}[fragile]{Methods}
  \begin{minted}[gobble = 4, frame = lines, linenos, breaklines]{rust}
    impl Hacker {
      fn add_cve(&mut self, cve: u32) {
        self.cves.push(cve);
      }
      fn cves(&self) -> &Vec<u32> {
        &self.cves
      }
    }
  \end{minted}

  \note{

    Таким образом задаются методы. На вход кроме обычных параметров подаётся
    \texttt{self}, т. е. мы подаём экземпляр самого объекта. Вот здесь типичный
    C++. У нас появляются ассоциированные со структурой методы.

  }

\end{frame}