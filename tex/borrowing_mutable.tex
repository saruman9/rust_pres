{

  \setbeamercolor{background canvas}{bg=}%
  \includepdf[pages = 128]{o_and_b.pdf}%

  \note {

    Следующая концепция --- \textbf{изменяемые заимствования}.

  }

}

{

  \setbeamercolor{background canvas}{bg=}%
  \includepdf[pages = 129-130]{o_and_b.pdf}%

  \note {

    Всё то же самое, что и в обычных заимствованиях, только функция
    \textbf{может изменять} данные.

  }

}

{

  \setbeamercolor{background canvas}{bg=}%
  \includepdf[pages = 131]{o_and_b.pdf}%

  \note {

    Всё то же самое. Только объект мы делаем таким, чтобы можно было его
    \textbf{изменять}.

  }

}

{

  \setbeamercolor{background canvas}{bg=}%
  \includepdf[pages = 133]{o_and_b.pdf}%

  \note {

    \textbf{Тип параметра} также следует поменять на \textbf{изменяемую ссылку}.

  }

}

{

  \setbeamercolor{background canvas}{bg=}%
  \includepdf[pages = 134]{o_and_b.pdf}%

  \note {

    Для передачи изменяемой ссылки, мы также должны \textbf{явно указать}, что
    данная \textbf{ссылка является изменяемой}.

  }

}

{

  \setbeamercolor{background canvas}{bg=}%
  \includepdf[pages = 136]{o_and_b.pdf}%

  \note {

    Так же, как и раньше мы передаём ссылку, но в Rust при передачи изменяемой
    ссылки у хозяина \textbf{пропадает возможность читать и писать данные} по
    этой ссылке.

  }

}

{

  \setbeamercolor{background canvas}{bg=}%
  \includepdf[pages = 138]{o_and_b.pdf}%

  \note {

    Начинаем выполнения функции.

  }

}

{

  \setbeamercolor{background canvas}{bg=}%
  \includepdf[pages = 140]{o_and_b.pdf}%

  \note {

    Изменяем переданный нам объект.

  }

}

{

  \setbeamercolor{background canvas}{bg=}%
  \includepdf[pages = 142]{o_and_b.pdf}%

  \note {

    Заканчиваем выполнение функции.

  }

}

{

  \setbeamercolor{background canvas}{bg=}%
  \includepdf[pages = 143]{o_and_b.pdf}%

  \note {

    \textbf{Эксклюзивные права на запись пропадают}, так как изменяемая ссылка
    уничтожается.

  }

}

{

  \setbeamercolor{background canvas}{bg=}%
  \includepdf[pages = 145]{o_and_b.pdf}%

  \note {

    Дальше можем оперировать строкой как обычно.

  }

}

{

  \setbeamercolor{background canvas}{bg=}%
  \includepdf[pages = 146]{o_and_b.pdf}%

  \note {

    Например распечатать её (данные передаются по ссылке).

  }

}

{

  \setbeamercolor{background canvas}{bg=}%
  \includepdf[pages = 148]{o_and_b.pdf}%

  \note {

    Главная функция заканчивает своё выполнение.

  }

}

{

  \setbeamercolor{background canvas}{bg=}%
  \includepdf[pages = 149]{o_and_b.pdf}%

  \note {

    Данные как всегда --- удаляются.

  }

}

{

  \setbeamercolor{background canvas}{bg=}%
  \includepdf[pages = 150]{o_and_b.pdf}%

  \note {

    Подведём итог.

  }

}

{

  \setbeamercolor{background canvas}{bg=}%
  \includepdf[pages = 151]{o_and_b.pdf}%

  \note {

    \textbf{Владение}.

    Мы имеем \textbf{полный доступ к данным}, которые удалятся, когда будут не
    нужны.

  }

}

{

  \setbeamercolor{background canvas}{bg=}%
  \includepdf[pages = 152]{o_and_b.pdf}%

  \note {

    \textbf{Заимствование}.

    По ссылке мы можем \textbf{многократно читать, но не писать}.

  }

}

{

  \setbeamercolor{background canvas}{bg=}%
  \includepdf[pages = 153]{o_and_b.pdf}%

  \note {

    \textbf{Изменяемое заимствование}.

    По изменяемой ссылке мы можем \textbf{только писать}.

  }

}