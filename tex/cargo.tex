\begin{frame}{\insertsubsection}
  \begin{columns}
    \begin{column}{.5\textwidth}
      Best package manager!
      \begin{itemize}
      \item Create structure of project
      \item Check and update dependencies
      \item Check, build and compile project
      \item Search and install packages
      \item Compile and run examples
      \item Generate docs
      \item Compile and run tests (in docs too)
      \item Compile and run benchmarks
      \item etc
      \end{itemize}
    \end{column}
    \begin{column}{.4\textwidth}
      \includegraphics[width = \textwidth]{c_p_p_package_manager}
    \end{column}
  \end{columns}

  \note {

    \begin{itemize}
    \item Cargo создаёт каркас проекта
    \item Следит за зависимостями, обновляет устаревшие по запросу
    \item Проводит проверку, собирает и компилирует
    \item Ищет и устанавливает в систему проекты
    \item Компилирует и запускает примеры
    \item Генерирует документацию
    \item Компилирует и запускает тесты
    \item Компилирует и запускает бенчмарки
    \end{itemize}

    В отличие от многих других языков, в Rust пакетный менеджер
    \textbf{разрабатывался вместе с компилятором языка}. Поэтому Cargo ---
    \textbf{это стандарт}. \textbf{Отсутствие зоопарка систем сборки} позволяет
    снизить порог вхождения в чужие проекты за счёт одинаковой структуры
    проектов и единого унифицированного способа сборки зависимостей.

    npm (js), pip (python), maven (java), hex (elixir), bundler (ruby), nuget
    (.NET), CPAN (perl), cabal (haskell), OPAM (OCaml), Elm, cmake, make, gult
    --- отстой.
    
  }

\end{frame}