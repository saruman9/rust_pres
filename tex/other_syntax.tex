\begin{frame}{\insertsubsection}

  \begin{columns}
    \begin{column}{0.4\textwidth}
      \begin{itemize}
      \item<1-> Generics
      \item<2-> Traits
        \begin{itemize}
        \item<2-> as interfaces
        \item<2-> for code reuse
        \item<2-> for operator overloading
        \end{itemize}
      \item<2-> Trait objects
      \item<3-> Closures (\mintinline{rust}{|x| 2 * x})
      \item<4-> Common collections
      \item<4-> Smart pointers
      \item<5-> Polymorphism, encapsulation
      \item<5-> ...
      \end{itemize}
    \end{column}
    \begin{column}{0.6\textwidth}
      \only<1,2,4,5>{%
      \center%
      \includegraphics<1>[height = 0.9\textheight]{cup_of_t.jpg}
      \includegraphics<2>[height = 0.9\textheight]{simple_go.jpg}
      \includegraphics<4>[height = 0.9\textheight]{smart_pointers.jpg}
      \includegraphics<5>[width = \textwidth]{polymorphism_c.jpg}
      }%
    \end{column}
  \end{columns}

  \note<1>{

    Я про многое не смогу рассказать. Например, про параметризованные типы, в
    Rust называются дженериками.

  }

  \note<2>{

    Также не смогу рассказать про трейты (типажи), кстати, кто не знал, в C++
    тоже есть типажи, но они не такие, как в Rust. В Rust типажи больше похожи
    на typeclass из Haskell.

    В Rust \textbf{нет всех тех проблем}, о которых нам пришлось беспокоиться на
    С++. Мы можем больше не думать о том, \textbf{как что-то теряется, когда
      функция вызывается каким-то образом}, и какое влияние оказывает
    \textbf{виртуальная диспетчеризация на наш код}. В Rust \textbf{все работает
      в едином стиле}, независимо от типа. Таким образом \textbf{целый класс
      детских ошибок просто исчезает}.

  }

  \note<3> {

    Про замыкания, которые основаны на типажах. Весьма интересно реализованы,
    ведь GC нет и замыкания не похожи на замыкания из C++.

  }

  \note<4> {

    Про коллекции структур из стандартной библиотеки и их весьма эффективную
    реализацию. Про саму стандартную библиотеку как таковую.

    Про умные указатели не расскажу, хотя там всё очень интересно, учитывая, что
    Rust позволяет достаточно \textbf{просто создавать мультипоточные программы
      не без помощи умных указателей}.

  }

  \note<5>{

    Также не расскажу, как в Rust реализовывается полиморфизм и инкапсуляция.

    И многое другое, всё рассказать невозможно.

  }

\end{frame}
