\begin{frame}[fragile]{\insertsubsection}
  \begin{onlyenv}<1>
    \begin{minted}[gobble = 4, frame = lines, label = Rustup install,%
      breaklines]{shell}
      $ curl https://sh.rustup.rs -sSf | sh
    \end{minted}
  \end{onlyenv}

  \begin{onlyenv}<2>
    \begin{minted}[gobble = 4, frame = lines, label = Toolchain format, breaklines]{text}
    <channel>[-<date>][-<host>]

    <channel>       = stable|beta|nightly|<version>
    <date>          = YYYY-MM-DD
    <host>          = <target-triple>
  \end{minted}

  \begin{minted}[gobble = 4, frame = lines, label = Install nightly toolchain,%
      breaklines]{shell}
      $ rustup toolchain install nightly
    \end{minted}
  \end{onlyenv}

  \begin{onlyenv}<3>
  \begin{minted}[gobble = 4, frame = lines, label = Cross compile,%
      breaklines]{shell}
      $ rustup target add mips64el-unknown-linux-gnuabi64
      $ cargo build --target=mips64el-unknown-linux-gnuabi64
    \end{minted}
  \end{onlyenv}

  \begin{onlyenv}<4>
    \begin{itemize}
    \item \texttt{aarch64-apple-ios}
    \item \texttt{aarch64-fuchsia}
    \item \texttt{arm-unknown-linux-gnueabihf}
    \item \texttt{armv5te-unknown-linux-musleabi}
    \item \texttt{asmjs-unknown-emscripten}
    \item \texttt{i686-pc-windows-msvc}
    \item \texttt{powerpc-unknown-linux-gnu}
    \item \texttt{riscv32imac-unknown-none-elf}
    \item \texttt{sparcv9-sun-solaris}
    \item \texttt{wasm32-unknown-emscripten}
    \item \texttt{x86\_64-unknown-redox}
    \item ...
    \end{itemize}
  \end{onlyenv}

  \note<1>{

    Rustup --- это \textbf{инсталлятор Rust}, если можно так выразиться. Через
    него вы можете установить \textbf{компоненты языка} (компилятор, стандартную
    библиотеку, утилиты для разработки, тулчейны, контролировать версии
    компилятора).

    Вот такой простой командой вы можете установить rustup или через системный
    пакетный менеджер.

  }

  \note<2>{

    У Rust официально есть три версии тулчейна:
    \begin{itemize}
    \item stable
    \item beta
    \item nightly
    \end{itemize}

  }

  \note<3>{

    Для кросскомпиляции проекта надо всего две команды!
    
  }
  
  \note<4>{
    
    Существует многочисленная поддержка архитектур, ОС и библиотек, в том числе
    \textbf{bare metal устройств}.
    
    Кроме инструментов, про которые я рассказал, \textbf{есть и другие}, но на
    них уже нет времени.

  }

\end{frame}
