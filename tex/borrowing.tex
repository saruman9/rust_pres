{

  \setbeamercolor{background canvas}{bg=}%
  \includepdf[pages = 85]{o_and_b.pdf}%

  \note {

    Концепция \textbf{заимствования}.

    Разделяемые заимствования.

    У нас как всегда \textbf{есть книга дракона}.

  }

}

{

  \setbeamercolor{background canvas}{bg=}%
  \includepdf[pages = 86]{o_and_b.pdf}%

  \note {

    Мы \textbf{отдаём} эту книгу, но \textbf{с условием, что нам её вернут}.

  }

}

{

  \setbeamercolor{background canvas}{bg=}%
  \includepdf[pages = 87]{o_and_b.pdf}%

  \note {

    \textbf{После того}, как нашу книгу \textbf{прочитали}, нам её
    \textbf{возвращают}.

    В отличие от передачи прав владения, мы даём книгу только \textbf{на время}.

  }

}

{

  \setbeamercolor{background canvas}{bg=}%
  \includepdf[pages = 88]{o_and_b.pdf}%

  \note {

    Теперь рассмотрим пример кода.

    У нас всё так же есть переменная \texttt{name} типа \texttt{String}.

  }

}

{

  \setbeamercolor{background canvas}{bg=}%
  \includepdf[pages = 90]{o_and_b.pdf}%

  \note {

    Обратите внимание, как поменялся тип параметра --- теперь передаётся не
    объект, а ссылка на объект, то бишь ссылка на переменную типа
    \texttt{String}.

  }

}

{

  \setbeamercolor{background canvas}{bg=}%
  \includepdf[pages = 92]{o_and_b.pdf}%

  \note {

    Чтобы передать нашу строку функции, \textbf{нам нужно получить ссылку},
    делаем это.

  }

}

{

  \setbeamercolor{background canvas}{bg=}%
  \includepdf[pages = 96]{o_and_b.pdf}%

  \note {

    Вызываем функцию \texttt{helper} и \textbf{передаём ей ссылку} на строку в
    качестве аргумента.

    Стоить заметить, что \textbf{ссылка на объект также остаётся у нас} в
    пользовании.

  }

}

{

  \setbeamercolor{background canvas}{bg=}%
  \includepdf[pages = 97]{o_and_b.pdf}%

  \note {

    Функция выполняется.

  }

}

{

  \setbeamercolor{background canvas}{bg=}%
  \includepdf[pages = 98]{o_and_b.pdf}%

  \note {

    Функция заканчивает своё выполнение.

  }

}

{

  \setbeamercolor{background canvas}{bg=}%
  \includepdf[pages = 99]{o_and_b.pdf}%

  \note {

    Функция исчезает, а с ней \textbf{исчезает и ссылка}.

  }

}

{

  \setbeamercolor{background canvas}{bg=}%
  \includepdf[pages = 100]{o_and_b.pdf}%

  \note {

    Мы снова можем вызвать функцию \texttt{helper}, передав ссылку на объект.

  }

}

{

  \setbeamercolor{background canvas}{bg=}%
  \includepdf[pages = 102]{o_and_b.pdf}%

  \note {

    Когда главная \textbf{функция прекращает своё выполнение}, то \textbf{и
      ссылка, и объект уничтожаются}.

  }

}

{

  \setbeamercolor{background canvas}{bg=}%
  \includepdf[pages = 103]{o_and_b.pdf}%

  \note {

    Стоит сказать, что в Rust \textbf{все объекты}, будь то значение или ссылки
    --- \textbf{неизменяемые} по умолчанию.

  }

}

{

  \setbeamercolor{background canvas}{bg=}%
  \includepdf[pages = 104]{o_and_b.pdf}%

  \note {

    Поэтому, если мы просто \textbf{читаем данные}, переданные нам по ссылке, то
    всё будет \textbf{хорошо}.

  }

}

{

  \setbeamercolor{background canvas}{bg=}%
  \includepdf[pages = 105]{o_and_b.pdf}%

  \note {

    А если попробуем \textbf{изменить} данные, то получим \textbf{ошибку}.

  }

}

{

  \setbeamercolor{background canvas}{bg=}%
  \includepdf[pages = 106]{o_and_b.pdf}%

  \note {

    Ошибку \textbf{компиляции}.

  }

}

{

  \setbeamercolor{background canvas}{bg=}%
  \includepdf[pages = 107]{o_and_b.pdf}%

  \note {

    На самом деле, данные \textbf{можно изменять}, но \textbf{только
      контролируемо}, об этом я не буду рассказывать подробно, считайте, что
    \textbf{по умолчанию всё неизменяемое}.

  }

}

{

  \setbeamercolor{background canvas}{bg=}%
  \includepdf[pages = 108]{o_and_b.pdf}%

  \note {

    Побольше рассмотрим ссылки.

  }

}

{

  \setbeamercolor{background canvas}{bg=}%
  \includepdf[pages = 109]{o_and_b.pdf}%

  \note {

    Всё тот же пример.

  }

}

{

  \setbeamercolor{background canvas}{bg=}%
  \includepdf[pages = 110]{o_and_b.pdf}%

  \note {

    На самом деле тип \texttt{String} представляет из себя структуру, которая
    состоит из нескольких полей.

  }

}

{

  \setbeamercolor{background canvas}{bg=}%
  \includepdf[pages = 110]{o_and_b.pdf}%

  \note {

    В функцию \texttt{helper} мы можем \textbf{передать часть строки}, как и в
    любых других ЯП.

    Но стоит обратить внимание на то, что в отличие от других ЯП, копирование
    объекта не происходит, мы передаём данные по ссылке (zero-cost abstraction).

  }

}

{

  \setbeamercolor{background canvas}{bg=}%
  \includepdf[pages = 112]{o_and_b.pdf}%

  \note {

    Если вы внимательны, то могли заметить, что \textbf{тип параметра у функции
      поменялся} на \texttt{\&str}.

  }

}

{

  \setbeamercolor{background canvas}{bg=}%
  \includepdf[pages = 113]{o_and_b.pdf}%

  \note {

    Так произошло по причине того, что мы передаём ссылку не на саму строку типа
    \texttt{String}, а только на \textbf{часть строки}.

  }

}

{

  \setbeamercolor{background canvas}{bg=}%
  \includepdf[pages = 114]{o_and_b.pdf}%

  \note {

    Таким образом у нас появляется новый объект типа \texttt{str}.

  }

}

{

  \setbeamercolor{background canvas}{bg=}%
  \includepdf[pages = 116]{o_and_b.pdf}%

  \note {

    Итак выполняем нашу функцию.

  }

}

{

  \setbeamercolor{background canvas}{bg=}%
  \includepdf[pages = 117]{o_and_b.pdf}%

  \note {

    Итак выполняем нашу функцию.

  }

}

{

  \setbeamercolor{background canvas}{bg=}%
  \includepdf[pages = 118]{o_and_b.pdf}%

  \note {

    После окончания выополнения функции, ссылка на объект удаляется, так же, как
    и объект типа \texttt{str}.

    \textbf{Вызываем повторно} нашу функцию, но уже со ссылкой на
    \texttt{String}. Стоит заметить, что \textbf{тип функции нам менять не
      нужно}.

  }

}

{

  \setbeamercolor{background canvas}{bg=}%
  \includepdf[pages = 119]{o_and_b.pdf}%

  \note {

    После выполнения все данные очищаются.

  }

}

{

  \setbeamercolor{background canvas}{bg=}%
  \includepdf[pages = 121]{o_and_b.pdf}%

  \note {

    Таким образом, мы можем выполнять высокоуровневый код с нулевой стоимостью.

  }

}

{

  \setbeamercolor{background canvas}{bg=}%
  \includepdf[pages = 122]{o_and_b.pdf}%

  \note {

    Допустим, мы разбиваем строку на слова, между которыми стоят пробелы.

    Вместо того, чтобы копировать каждое слово, мы передаём ссылку на объект.

  }

}

{

  \setbeamercolor{background canvas}{bg=}%
  \includepdf[pages = 123]{o_and_b.pdf}%

  \note {

    Пример.

  }

}

{

  \setbeamercolor{background canvas}{bg=}%
  \includepdf[pages = 124-126]{o_and_b.pdf}%

  \note {

    Пример.

  }

}