\begin{frame}[fragile]{\insertsubsection}

  \center%
  \includegraphics<1>[height = .8\textheight]{compiler_driven.png}%

  \begin{onlyenv}<2>
    \begin{minted}[gobble = 4, frame = lines, framesep = 7pt, linenos,
      breaklines, label = C++]{text}
    Too big, too unclear
  \end{minted}

  \begin{minted}[gobble = 4, frame = lines, framesep = 7pt, linenos, breaklines,
    label = Rust]{text}
    error: expected type, found `'static`
     --> test_err.rs:3:9
      |
    3 |     Ref('static str),
      |         ^^^^^^^

    error: aborting due to previous error
  \end{minted}
  \end{onlyenv}

  \begin{onlyenv}<3>
    \begin{minted}[gobble = 4, frame = lines, framesep = 7pt, linenos,
      breaklines, label = C++]{text}
    In file included from /usr/include/c++/8.2.1/cassert:44,
                     from test_err.cpp:3:
    test_err.cpp: In function ‘int main()’:
    test_err.cpp:17:5: error: invalid use of void expression
         assert(std::holds_alternative<std::string>(y)); // succeeds
     ^~~~~~
  \end{minted}
  \end{onlyenv}

  \begin{onlyenv}<4>
    \begin{minted}[gobble = 4, frame = lines, framesep = 7pt, linenos,
      breaklines, label = Rust]{text}
    error: expected one of `.`, `;`, `?`, or an operator, found `}`
     --> test_err.rs:6:1
      |
    5 |     let y = S("xyz".to_string())
      |                                 - expected one of `.`, `;`, `?`, or an operator here
    6 | }
      | ^ unexpected token

    error: aborting due to previous error
  \end{minted}
  \end{onlyenv}

  \note<1>{

    Есть много различных методологий разработки (через тестирование, через
    поведение, agile), но у меня впервые разработка на основе поведения
    компилятора (или вывода на ошибках).

  }

  \note<2>{

    Пример работы с параметрическими типами. Я просто убрал символ ссылки.

  }

  \note<3>{

    Убрал точку с запятой.

  }

  \note<4>{

    Кроме того, что компилятор говорит точно, где ошибка, он ещё может
    посоветовать \textbf{как её исправить}.

    Кроме всего, почти для каждой ошибки есть раздел в \textbf{одной большой
      книге ошибок}, где описывается типичный пример ошибки и \textbf{гайд по её
      исправлению}.

  }


\end{frame}