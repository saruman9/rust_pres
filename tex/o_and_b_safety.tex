{

  \setbeamercolor{background canvas}{bg=}%
  \includepdf[pages = 155]{o_and_b.pdf}%

  \note {

    Как же в итоге все эти концепции влияют на безопасность?

  }

}

{

  \setbeamercolor{background canvas}{bg=}%
  \includepdf[pages = 156]{o_and_b.pdf}%

  \note {

    Итак, у нас есть программа.

  }

}

{

  \setbeamercolor{background canvas}{bg=}%
  \includepdf[pages = 157]{o_and_b.pdf}%

  \note {

    У нас создаётся \textbf{неинициализированная} переменная \texttt{r}.

  }

}

{

  \setbeamercolor{background canvas}{bg=}%
  \includepdf[pages = 158]{o_and_b.pdf}%

  \note {

    Данная переменная, соответственно, \textbf{создаётся на стеке}.

  }

}

{

  \setbeamercolor{background canvas}{bg=}%
  \includepdf[pages = 160]{o_and_b.pdf}%

  \note {

    Создаём ещё одну переменную \texttt{name} типа \texttt{String}. Переменная
    создаётся \textbf{в своей области видимости}, позже я про это расскажу.

    Стоит также заметить, что в Rust \textbf{есть type inference (вывод типов)},
    который ранее был характерен в основном для функциональных ЯП.

  }

}

{

  \setbeamercolor{background canvas}{bg=}%
  \includepdf[pages = 161]{o_and_b.pdf}%

  \note {

    Как мы видим, новая переменная также помещается на стек.

  }

}

{

  \setbeamercolor{background canvas}{bg=}%
  \includepdf[pages = 163]{o_and_b.pdf}%

  \note {

    Производим присвоение переменной \texttt{r} ссылки на объект \texttt{name}.

  }

}

{

  \setbeamercolor{background canvas}{bg=}%
  \includepdf[pages = 164]{o_and_b.pdf}%

  \note {

    На стеке это будет выглядеть примерно так.

  }

}

{

  \setbeamercolor{background canvas}{bg=}%
  \includepdf[pages = 166]{o_and_b.pdf}%

  \note {

    Продолжаем выполнение...

  }

}

{

  \setbeamercolor{background canvas}{bg=}%
  \includepdf[pages = 167]{o_and_b.pdf}%

  \note {

    ...и \textbf{выходим из области видимости} переменной \texttt{name}.

  }

}

{

  \setbeamercolor{background canvas}{bg=}%
  \includepdf[pages = 169]{o_and_b.pdf}%

  \note {

    Пытаемся распечатать содержимое переменной \texttt{r}.

  }

}

{

  \setbeamercolor{background canvas}{bg=}%
  \includepdf[pages = 170]{o_and_b.pdf}%

  \note {

    Упс! Висячая ссылка!

  }

}

{

  \setbeamercolor{background canvas}{bg=}%
  \includepdf[pages = 172]{o_and_b.pdf}%

  \note {

    Давайте посмотрим, как Rust с этим разбирается.

  }

}

{

  \setbeamercolor{background canvas}{bg=}%
  \includepdf[pages = 174]{o_and_b.pdf}%

  \note {

    В Rust есть понятие \textbf{время жизни ссылки} --- это промежутки в коде,
    где ссылка используется.

  }

}

{

  \setbeamercolor{background canvas}{bg=}%
  \includepdf[pages = 176]{o_and_b.pdf}%

  \note {

    Сначала ссылка присваивается переменной \texttt{r}.

  }

}

{

  \setbeamercolor{background canvas}{bg=}%
  \includepdf[pages = 177]{o_and_b.pdf}%

  \note {

    Переменная \texttt{r}, объявленная выше, в свою очередь используется при
    выводе в терминал.

  }

}

{

  \setbeamercolor{background canvas}{bg=}%
  \includepdf[pages = 180]{o_and_b.pdf}%

  \note {

    Вся эта область и будет являться временем жизни ссылки.

  }

}

{

  \setbeamercolor{background canvas}{bg=}%
  \includepdf[pages = 181]{o_and_b.pdf}%

  \note {

    Есть ещё одно понятие --- \textbf{область видимости}, область, где созданные
    данные могут быть использованы.

  }

}

{

  \setbeamercolor{background canvas}{bg=}%
  \includepdf[pages = 182]{o_and_b.pdf}%

  \note {

    Вот область видимости переменной \texttt{name}.

  }

}

{

  \setbeamercolor{background canvas}{bg=}%
  \includepdf[pages = 183]{o_and_b.pdf}%

  \note {

    Производя сравнения времени жизни переменной с областью видимости ссылки,
    Rust приходит к выводу, что \textbf{здесь ошибка (на этапе компиляции)}.

  }

}

{

  \setbeamercolor{background canvas}{bg=}%
  \includepdf[pages = 184]{o_and_b.pdf}%

  \note {

    Посмотрим, как Rust справляется с ошибками, возникающими \textbf{при работе
      с потоками}.

  }

}

{

  \setbeamercolor{background canvas}{bg=}%
  \includepdf[pages = 185]{o_and_b.pdf}%

  \note {

    У нас есть параметр функции \texttt{name}, оперировать которым мы можем
    только в области видимости функции, т. е. время жизни ссылки будет ---
    область видимости данной функции.

  }

}

{

  \setbeamercolor{background canvas}{bg=}%
  \includepdf[pages = 188]{o_and_b.pdf}%

  \note {

    Если мы создадим поток и будем использовать переменную \texttt{name} во
    вновь созданном потоке, то мы должны понимать, что \textbf{поток будет
      исполняться вне данной функции}.

  }

}

{

  \setbeamercolor{background canvas}{bg=}%
  \includepdf[pages = 189]{o_and_b.pdf}%

  \note {

    Что приведёт к ошибке --- использование ссылки, которая возможно \textbf{уже
      не существует}. Rust нам об этом скажет уже на этапе компиляции.

  }

}

{

  \setbeamercolor{background canvas}{bg=}%
  \includepdf[pages = 191]{o_and_b.pdf}%

  \note {

    Мы можем задать \textbf{статическое время жизни ссылки} --- это значит, что
    ссылка будет жить, пока программа не закончит своё выполнение.

  }

}

{

  \setbeamercolor{background canvas}{bg=}%
  \includepdf[pages = 194]{o_and_b.pdf}%

  \note {

    Переходим к следующим возможным ошибкам --- изменение используемых данных.

  }

}

{

  \setbeamercolor{background canvas}{bg=}%
  \includepdf[pages = 196]{o_and_b.pdf}%

  \note {

    У нас создаётся \textbf{изменяемая} переменная типа \texttt{String}.

  }

}

{

  \setbeamercolor{background canvas}{bg=}%
  \includepdf[pages = 199]{o_and_b.pdf}%

  \note {

    Затем мы создаём ссылку на первый элемент строки.

  }

}

{

  \setbeamercolor{background canvas}{bg=}%
  \includepdf[pages = 202]{o_and_b.pdf}%

  \note {

    Потом нам вздумалось изменить строку.

  }

}

{

  \setbeamercolor{background canvas}{bg=}%
  \includepdf[pages = 203]{o_and_b.pdf}%

  \note {

    Добавили в конец \texttt{s}.

  }

}

{

  \setbeamercolor{background canvas}{bg=}%
  \includepdf[pages = 204]{o_and_b.pdf}%

  \note {

    Первоначальные данные с кучи у нас пропадут, т. к. мы, возможно, сделали
    реаллокацию.

  }

}

{

  \setbeamercolor{background canvas}{bg=}%
  \includepdf[pages = 206]{o_and_b.pdf}%

  \note {

    Что будет, если мы попытаемся вывести содержимое по ссылке \texttt{slice}?

  }

}

{

  \setbeamercolor{background canvas}{bg=}%
  \includepdf[pages = 207]{o_and_b.pdf}%

  \note {

    Верно! Снова ошибка висячей ссылки.

  }

}

{

  \setbeamercolor{background canvas}{bg=}%
  \includepdf[pages = 209]{o_and_b.pdf}%

  \note {

    Как же Rust поступает в данном случае?

    Во время компиляции проверяется:

    \begin{itemize}
    \item Если есть одна \textbf{ссылка на чтение}, то:
      \begin{itemize}
      \item создание ещё ссылок \textbf{на чтение ошибок не вызовет};
      \item создание \textbf{ссылок на изменение запрещено};
      \item данные правила будут работать до тех пор, \textbf{пока жива ссылка}.
      \end{itemize}
    \item Если есть одна \textbf{ссылка на изменение}, то:
      \begin{itemize}
      \item у неё \textbf{эксклюзивные права}, никаких новых ссылок быть не
        может;
      \item данные правила будут работать до тех пор, пока \textbf{жива ссылка}.
      \end{itemize}
    \end{itemize}

    Таким образом, у нас ни при каких обстоятельствах \textbf{не может быть
      ссылок на чтение и изменение одновременно}.

  }

}

{

  \setbeamercolor{background canvas}{bg=}%
  \includepdf[pages = 210]{o_and_b.pdf}%

  \note {

    Возвращаемся к нашему примеру с висячей ссылкой, полученной в результате
    изменения объекта.

  }

}

{

  \setbeamercolor{background canvas}{bg=}%
  \includepdf[pages = 212]{o_and_b.pdf}%

  \note {

    Когда мы создаём ссылку на чтение, мы автоматически \textbf{блокируем
      создание ссылок на изменение}.

  }

}

{

  \setbeamercolor{background canvas}{bg=}%
  \includepdf[pages = 213]{o_and_b.pdf}%

  \note {

    В данной области не может быть создано ссылок на изменение.

  }

}

{

  \setbeamercolor{background canvas}{bg=}%
  \includepdf[pages = 214]{o_and_b.pdf}%

  \note {

    При вызове метода \texttt{push\_str} \textbf{создаётся изменяемая ссылка}, а
    это запрещено, результат --- \textbf{ошибка на этапе компиляции}.

  }

}

{

  \setbeamercolor{background canvas}{bg=}%
  \includepdf[pages = 216]{o_and_b.pdf}%

  \note {

    Рассмотрим немного другой пример.

  }

}

{

  \setbeamercolor{background canvas}{bg=}%
  \includepdf[pages = 217]{o_and_b.pdf}%

  \note {

    Здесь также создаётся ссылка на чтение.

  }

}

{

  \setbeamercolor{background canvas}{bg=}%
  \includepdf[pages = 218]{o_and_b.pdf}%

  \note {

    Но время жизни ссылки уже другое.

  }

}

{

  \setbeamercolor{background canvas}{bg=}%
  \includepdf[pages = 219]{o_and_b.pdf}%

  \note {

    В данной области у нас \textbf{нет прав создавать ссылку на изменение}.

  }

}

{

  \setbeamercolor{background canvas}{bg=}%
  \includepdf[pages = 220]{o_and_b.pdf}%

  \note {

    Но \textbf{за границами жизни ссылки} на чтение мы \textbf{вправе создавать
      ссылки} какие захотим.

  }

}