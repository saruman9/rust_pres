\begin{frame}[fragile]{NULL dereferences}

  \begin{minted}[gobble = 4, frame = lines, label = C, linenos]{c}
    uint8_t* pointer = (uint8_t*) malloc(SIZE); // Might return NULL
    for(int i = 0; i < SIZE; ++i) {
      pointer[i] = i; // Might cause a Segmentation Fault
    }
  \end{minted}
    
  \begin{minted}[gobble = 4, frame = lines, label = Rust, linenos]{rust}
    let mut vec = vec![0 as u8; SIZE];
    for i in 0..SIZE { // As C code
      vec[i] = i;
    }
  \end{minted}

  \note {

    Как мы можем наблюдать, в C коде мы можем получить \textbf{SEGFAULT на
      каждом шагу}, т. к. \texttt{malloc} и другие функции могут вернуть
    \texttt{null}.

    Если же писать на Rust в стиле, похожем на C, то в данной версии мы не можем
    получить \texttt{null} априори, т. к. \textbf{\texttt{null} в Rust нет}.

    В случае, \textbf{если аллокация памяти будет неудачной}, то мы либо получим
    \textbf{панику}, либо \textbf{сможем отловить ошибку}.
    
  }

\end{frame}

\begin{frame}[fragile]{NULL dereferences (continue)}
  \begin{minted}[gobble = 4, frame = lines, label = Functional Rust, linenos]%
    {rust}
    let vec: Vec<u8> = (0..10).collect();
  \end{minted}

  \begin{minted}[gobble = 4, frame = lines, label = Rust References, linenos]%
    {rust}
    let my_var: u32 = 42;
    let my_ref: &u32 = &my_var; // References ALWAYS point
                                // to valid data
    let my_var2 = *my_ref; // An example for a Dereference
  \end{minted}

  \note{

    В функциональном стиле всё ещё проще и без ошибок.

    Если же работать с ссылками, то мы также \textbf{никогда не получим NULL},
    потому что \textbf{в Rust нет NULL} вообще, в отличие от других языков.

  }

\end{frame}

\begin{frame}[fragile]{Use after free (C)}

  \begin{minted}[gobble = 4, frame = lines, label = C, linenos]{c}
    uint8_t* pointer = (uint8_t*) malloc(SIZE);
    // ...
    if (err) {
      abort = 1;
      free(pointer);
    }
    // ...
    if (abort) {
      logError("operation aborted", pointer);
    }
  \end{minted}

  \note {

    Сначала мы \textbf{выделяем память} на куче, затем \textbf{освобождаем
      память}, потом пытается \textbf{читать память}, которая уже освобождена.

    \textbf{Поведение будет неоднозначным}, возможно, мы даже не заметим ошибки
    до поры до времени.

  }

\end{frame}

\begin{frame}[fragile]{Use after free (Rust)}
    
  \begin{minted}[gobble = 4, frame = lines, label = Rust, linenos]{rust}
    let vec: Vec<u32> = Vec::new();
    {
      {
        let vec_1 = vec; // vec's ownership has been moved
      } // the Vec will be freed (dropped) here
    }
  \end{minted}

  \note {

    В Rust, как и в C++ \textbf{работает RAII} (получение ресурса есть
    инициализация), поэтому ошибок UAF в Rust быть не может. Кроме этого, как
    говорилось выше, работают \textbf{и другие правила}, например, концепция
    владения.

  }

\end{frame}

\begin{frame}[fragile]{Dangling pointers (C)}
  \begin{minted}[gobble = 4, frame = lines, label = C, linenos]{c}
    uint8_t* get_dangling_pointer(void) {
      uint8_t array[4] = {0};
      return &array[0];
    }
  \end{minted} 

  \note {

    Создаём в функции переменную на стеке, затем возвращаем ссылку на эту
    переменную --- у нас висячая ссылка. Ссылка указывает на неизвестные данные
    на стеке.
    
  }
\end{frame}

\begin{frame}[fragile]{Dangling pointers (Rust)}

  \begin{minted}[gobble = 4, frame = lines, label = Rust, linenos]{rust}
    fn get_dangling_pointer() -> &u8 {
      let array = [0; 4];
      &array[0]
    }
  \end{minted}
  
  \begin{minted}[gobble = 4, frame = lines, label = Compile time error, breaklines]{text}
      |
    1 | fn get_dangling_pointer() -> &u8 {
      |                              ^ help: consider giving it a 'static lifetime: `&'static`
      |
      = help: this function's return type contains a borrowed value, but there is no value for it to be borrowed from
  \end{minted}

  \note {

    В Rust при попытке вернуть ссылку на переменные, находящиеся на стеке,
    возникнет ошибка при компиляции.

    Компилятор советует указать статическое время жизни ссылки для того, чтобы
    она была доступна при выходе из функции.

  }

\end{frame}

\begin{frame}[fragile]{Out of bounds (C)}
  \begin{minted}[gobble = 4, frame = lines, label = C, linenos, breaklines]{c}
    void print_out_of_bounds(void) {
      uint8_t array[4] = {0};
      printf("%u\r\n", array[4]);
    }
    // prints memory that's outside `array` (on the stack)
  \end{minted}

  \note {

    При попытке получить значение вне массива, мы получим случайное значение на
    стеке.
    
  }
\end{frame}

\begin{frame}[fragile]{Out of bounds (Rust)}
  \begin{minted}[gobble = 4, frame = lines, label = Rust, linenos, breaklines]{rust}
    fn print_panics() {
      let array = [0; 4];
      println!("{}", array[4]);
    }
  \end{minted}

  \begin{minted}[gobble = 4, frame = lines, label = Compile time error, breaklines]{text}
    error: index out of bounds: the len is 4 but the index is 4
     --> test.rs:8:20
      |
    3 |     println!("{}", array[4]);
      |                    ^^^^^^^^
      |
      = note: #[deny(const_err)] on by default
  \end{minted}

  \note {

    В случае с Rust, если существует возможность на стадии компиляции
    \textbf{определить границы массива}, то ошибка \textbf{будет отловлена уже
      на стадии компиляции}.

    Если же компилятор \textbf{не может определить}, по какому индексу
    происходит чтение, то в случае чтение за границами массива возникнет ошибка
    во время исполнения --- \textbf{паника}.
    
  }
\end{frame}