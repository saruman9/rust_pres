\begin{frame}[fragile]{\insertsubsection}

  \begin{minted}[gobble = 4, frame = lines, framesep = 7pt, linenos,
    breaklines]{rust}
    //! # Main
    //! Module docs

    /// Docs
    // Comments
    fn main() {
        let x = 31337;
        println!("The value of x is: {}", x); // 31337
        let mut y: u8 = 5;
        y = x as u8;
        println!("The value of y is: {}", y); // 105
    }
  \end{minted}

  \note {

    Есть \textbf{встроенная система документации}, о которой расскажу позднее.

    От функциональных языков в Rust пришла \textbf{иммутабельность
      по-умолчанию}: все переменные, объекты, функции и т. п. по-умолчанию
    неизменяемые.

    Rust обладает \textbf{статической сильной типизацией} с возможностью
    \textbf{вывода типов} (type inference). Используется \textbf{система типов
      Хиндли--Милнера} (впервые использовалась в ML языках), благодаря которой и
    работает вывод типов, а также другие особенности, о которых позже.

    \textbf{Приведение типов} в Rust автоматическое для безопасных случаев, в
    остальных --- ручной кастинг двух видов \texttt{as} и \texttt{transmute}.

  }

\end{frame}

\begin{frame}[fragile]{\insertsubsection}

  \begin{minted}[gobble = 4, frame = lines, framesep = 7pt, linenos,
    breaklines]{rust}
    fn nsa(is_hack: bool, backdoor: &str, blue_pill: String) -> f64 {
        for c in blue_pill.chars() {
            print!("{}", c);
        }
        if is_hack {
            loop { break 3.1337; }
        } else if backdoor.len() > 3 {
            42.0 - 42.0
        } else {
            3.14
        }
    }
  \end{minted}

  \note {

    В Rust \textbf{всё есть выражение} (expression).

    Пример задания параметров функции и возвращаемого значения.

    Для управления потоком используются: \texttt{while}, \texttt{loop},
    \texttt{if-else}, \texttt{for} --- проходит по элементам, например, по
    итераторам.

  }

\end{frame}