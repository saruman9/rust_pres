\begin{frame}{\insertsubsection}
  \begin{columns}
    \begin{column}{0.6\textwidth}

      \textbf{Originally:} Rust had message passing built into the language

      \textbf{Now:} library-based, multi-paradigm

      \begin{itemize}
      \item rayon (parallel processing, thread pool)
      \item tokio, futures (I/O, async)
      \item coroutine, coio (coroutine)
      \item crossbeam, mio (low-level concurrency)
      \end{itemize}

      Libraries leverage \textbf{ownership and traits} to avoid data races
    \end{column}
    \begin{column}{0.4\textwidth}
      \includegraphics[width = 0.8\textwidth]{go_conc.jpg}
    \end{column}
  \end{columns}

  \note {

    На \textbf{уровне стандартной библиотеки} Rust имеет реализацию параллелизма
    \textbf{в виде передачи сообщений}, также имеются \textbf{вспомогательные
      типа}: мьютексы, атомарные типы и прочее, сейчас \textbf{на стадии обкатки
      async-await}.

    Но существует множество \textbf{решений в виде библиотек}, которые
    реализовывают параллелизм в виде \textbf{множества различных парадигм}. Все
    библиотеки \textbf{лишены багов типа гонка данных}, благодаря использованию
    внутри концепций \textbf{владения и типажей}.
    
  }
\end{frame}

\begin{frame}[fragile]{Concurrent quicksort}
  \begin{minted}[gobble = 4, frame = lines, label = Rust, linenos, breaklines]{rust}
    fn qsort(vec: &mut [i32]) {
      if vec.len() <= 1 { return; }
      let pivot = vec[random(vec.len())];
      let mid = vec.partition(vec, pivot);
      let (less, greater) = vec.split_at_mut(mid);

      rayon::join(|| qsort(less),
                  || qsort(greater));

    }
  \end{minted}

  \note {

    Кто и как часто писал распараллеленные программы? Я писал очень мало.

    Не требуется понимать весь код, который представлен здесь. Достаточно
    понять, как \textbf{легко с помощью библиотеки} \texttt{rayon}
    (среднеуровневая библиотека) здесь реализована \textbf{распараллеленная
      быстрая сортировка}.

    На \textbf{других ЯП} распараллеленная быстрая сортировка, как правило,
    выглядит гораздо \textbf{сложнее}, в Rust же она \textbf{лишена таких багов,
      как data races}, благодаря вышеназванным концепциям.

    \textbf{Больше о параллелизме я говорить не буду}, так как данная тема
    займёт всё время, да и я не шарю на столько, чтобы про него вещать.
    
  }
\end{frame}